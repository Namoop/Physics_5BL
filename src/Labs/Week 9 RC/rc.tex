%! suppress = EscapeHashOutsideCommand
%! Author = Theodore Capinski
%! Date = 3/13/2024

% Preamble
\documentclass[11pt]{article}
\let\oldsection\section
\renewcommand\section{\clearpage\oldsection}
\setcounter{section}{-1}
\counterwithin{figure}{section}

% Packages
\usepackage{amsmath}
\usepackage{hyperref}
\usepackage{graphicx}
\usepackage{tikz}
\usepackage{indentfirst}
\usepackage{circuitikz}
\usepackage{calc}

%%%%%%%%%%%%%%%%%%%%%%%%%%%%%%%%%%%%%%%%%%%%%%%%%%%%%%%%%%%%%%%%%%%%%%
% LaTeX Overlay Generator - Annotated Figures v0.0.1
% Created with http://ff.cx/latex-overlay-generator/
%%%%%%%%%%%%%%%%%%%%%%%%%%%%%%%%%%%%%%%%%%%%%%%%%%%%%%%%%%%%%%%%%%%%%%
%\annotatedFigureBoxCustom{bottom-left}{top-right}{label}{label-position}{box-color}{label-color}{border-color}{text-color}
\newcommand*\annotatedFigureBoxCustom[8]{\draw[#5,thick,rounded corners] (#1) rectangle (#2);\node at (#4) [fill=#6,thick,shape=circle,draw=#7,inner sep=2pt,font=\sffamily,text=#8] {\textbf{#3}};}
%\annotatedFigureBox{bottom-left}{top-right}{label}{label-position}
\newcommand*\annotatedFigureBox[4]{\annotatedFigureBoxCustom{#1}{#2}{#3}{#4}{white}{white}{black}{black}}
\newcommand*\annotatedFigureText[4]{\node[draw=none, anchor=south west, text=#2, inner sep=0, text width=#3\linewidth,font=\sffamily] at (#1){#4};}
\newenvironment {annotatedFigure}[1]{\centering\begin{tikzpicture}
                                                   \node[anchor=south west,inner sep=0] (image) at (0,0) { #1};\begin{scope}[x={(image.south east)},y={(image.north west)}]}{\end{scope}\end{tikzpicture}}
%%%%%%%%%%%%%%%%%%%%%%%%%%%%%%%%%%%%%%%%%%%%%%%%%%%%%%%%%%%%%%%%%%%%%%

\newcommand{\todo}[1]{\textcolor{red}{TODO: #1}\PackageWarning{TODO:}{#1!}}

\title{Physics 5BL Lab Report RC Circuits}
\author{T.~Capinski \and A.~Patel}

% Document
\begin{document}
    \maketitle
    \tableofcontents

    \section*{Introduction}\label{sec:introduction}
    \addcontentsline{toc}{section}{Introduction}
    this report will cover RC circuits.









    \section*{Theory}\label{sec:theory}









    \section{Part 1A: Measuring RC Time Constant}\label{sec:part1a_time_constant}
    \subsection{Methods}\label{subsec:part1a_methods}
    \subsection{Analysis}\label{subsec:part1a_analysis}

    \section{Part 1B: Handcrafting a Capacitor}\label{sec:part1b_capacitor}
    \subsection{Methods}\label{subsec:part1b_methods}
    \subsection{Analysis}\label{subsec:part1b_analysis}

    \section{Part 1 Conclusion}\label{sec:part1_conclusion}



    \section{Part 2A: RLC Underdamped Transient Response}\label{sec:part2a_underdamped}
    \subsection{Methods}\label{subsec:part2a_methods}
    \subsection{Analysis}\label{subsec:part2a_analysis}

    \section{Part 2B: RLC Overdamped Transient Response}\label{sec:part2b_overdamped}
    \subsection{Methods}\label{subsec:part2b_methods}
    \subsection{Analysis}\label{subsec:part2b_analysis}

    \section{Part 2 Conclusion}\label{sec:part2_conclusion}


    \section{Lab Conclusion}\label{sec:conclusion}


    \appendix
    \section{Data}\label{sec:data}

    \section{References}\label{sec:references}

    Lab Manual







\end{document}