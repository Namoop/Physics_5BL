%! suppress = EscapeHashOutsideCommand
%! Author = Theodore Capinski
%! Date = 3/5/2024

% Preamble
\documentclass[11pt]{article}
\let\oldsection\section
\renewcommand\section{\clearpage\oldsection}
\setcounter{section}{-1}
\counterwithin{figure}{section}

% Packages
\usepackage{amsmath}
\usepackage{hyperref}
\usepackage{graphicx}
\usepackage{tikz}
\usepackage{indentfirst}
\usepackage{circuitikz}

%%%%%%%%%%%%%%%%%%%%%%%%%%%%%%%%%%%%%%%%%%%%%%%%%%%%%%%%%%%%%%%%%%%%%%
% LaTeX Overlay Generator - Annotated Figures v0.0.1
% Created with http://ff.cx/latex-overlay-generator/
%%%%%%%%%%%%%%%%%%%%%%%%%%%%%%%%%%%%%%%%%%%%%%%%%%%%%%%%%%%%%%%%%%%%%%
%\annotatedFigureBoxCustom{bottom-left}{top-right}{label}{label-position}{box-color}{label-color}{border-color}{text-color}
\newcommand*\annotatedFigureBoxCustom[8]{\draw[#5,thick,rounded corners] (#1) rectangle (#2);\node at (#4) [fill=#6,thick,shape=circle,draw=#7,inner sep=2pt,font=\sffamily,text=#8] {\textbf{#3}};}
%\annotatedFigureBox{bottom-left}{top-right}{label}{label-position}
\newcommand*\annotatedFigureBox[4]{\annotatedFigureBoxCustom{#1}{#2}{#3}{#4}{white}{white}{black}{black}}
\newcommand*\annotatedFigureText[4]{\node[draw=none, anchor=south west, text=#2, inner sep=0, text width=#3\linewidth,font=\sffamily] at (#1){#4};}
\newenvironment {annotatedFigure}[1]{\centering\begin{tikzpicture}
                                                   \node[anchor=south west,inner sep=0] (image) at (0,0) { #1};\begin{scope}[x={(image.south east)},y={(image.north west)}]}{\end{scope}\end{tikzpicture}}
%%%%%%%%%%%%%%%%%%%%%%%%%%%%%%%%%%%%%%%%%%%%%%%%%%%%%%%%%%%%%%%%%%%%%%

\newcommand{\todo}[1]{\textcolor{red}{TODO: #1}\PackageWarning{TODO:}{#1!}}

\title{Physics 5BL Lab Report EM1}
\author{T.~Capinski \and A.~Patel}

% Document
\begin{document}
    \maketitle
    \tableofcontents

    \section*{Introduction}\label{sec:introduction}
    \addcontentsline{toc}{section}{Introduction}

    This report will cover the Ohm's Law lab. We will be employing tools such as the Digital Multimeter (DMM) and the IOLab apparatus. Through measurements of resistance, current, and voltage, we will measure Ohmic and non-Ohmic behaviors in circuit elements, distinguishing between components that adhere to Ohm's Law and those that exhibit nonlinear characteristics.

    In addition, the lab will incorporate the application of Kirchhoff's Circuit Rules, which are used for analyzing circuit configurations. By employing the Junction Rule and Loop Rule, we can gain insight into the conservation of charge and energy within electrical circuits.

    For experiment 0, we simply measured the resistance of a 1 $\Omega$ resistor, however, we found that our measurement was outside the expected range of values using .01 $\Omega$ as the error in the resistors measurement. 

    For experiment 1A, we first measured the transfer ratio (H ≡ Vout/Vin) of a circuit with 2 1000 $\Omega$ resistors in it, which we found to be .5005, inside the range of error for the accepted value of 0.5. we then used the DMM to measure voltage, comparing this to our voltage from the previous part, and finding that they both agreed with each other. For experiment 1B, we created a circuit with a 10,000 $\Omega$ resistor in it and measured the current using both the IOLab and the DMM. Our measured current from the IOLab was 0.229 mA, slightly out of the range of the accepted value of 0.2 mA. However, when using the DMM, we measured exactly 0.2 mA. We then determined the IOLab had higher sensitivity due to the large amount of decimals it gave and its high error from the expected value.

    For experiment 2A, we first measured the current across a 10,000 $\Omega$ resistor to measure ohmic behavior. We used 10 different input voltages and found that it obeyed ohms law. We then calculated the resistance of the resistor to be 9379.79 ± 233.97 Ω, which is slightly out of range of the accepted value. For experiment 2B, we measured the voltage across 2 LEDs to test for ohmic behavior. We used a 1000 $\Omega$ resistor in series to limit current and again took 10 data points for each LED. We found that both green and red LEDs did not obey ohms law, as they both needed a minimum voltage difference in order to let current through, then acted like wires. We also found the resistance of the resistor to be 996.50 ± 2.56 Ω when using the green LED and 989.35 ± 10.81 Ω when using the red LED, which were both inside the acceptable range of values. This is because the LED acted like a wire once the minimum voltage was met, so it allowed us to properly measure resistance.

    For experiment 3, we measured the current across different resistors in a circuit involving multiple batteries and resistors. Our data did not match the predicted values and we could not verify Kirchhoffs junction rule equation.
    
    \section*{Theory}\label{sec:theory}
    Ohm's law states that the current through a conductor between two points is directly proportional to the voltage across the two points.
    This relationship is represented by the equation
    \begin{equation}\label{eq:ohm_equation}
        V = I \cdot R
    \end{equation}
    where $V$ is the voltage, $I$ is the current, and $R$ is the resistance.
    Such a relationship is linear and is known as ohmic behaviour.
    To find the resistance of a material, we can use the equation
    \begin{equation}\label{eq:resistance_equation}
        R = \frac{V}{I}
    \end{equation}

    As covered in the report overview, resistors in series can be summed, such as
    \begin{equation}\label{eq:resistors_in_series}
        R_{\text{total}} = R_1 + R_2 + \ldots
    \end{equation}
    and when in parallel, the reciprocals of the resistances can be summed, such as
    \begin{equation}\label{eq:resistors_in_parallel}
        \frac{1}{R_{\text{total}}} = \frac{1}{R_1} + \frac{1}{R_2} + \ldots
    \end{equation}

    To find the transfer ratio, we can use the input and output voltages, seen below:
    \begin{equation}\label{eq:transfer_ratio}
        H = \frac{V_{\text{in}}}{V_{\text{out}}}
    \end{equation}

    Finally, Kirchhoff's Laws are used to analyze complex circuits.
    The first law states that the sum of currents entering a junction is equal to the sum of currents leaving the junction.
    The second law states that the sum of the voltage drops around a closed loop is equal to the voltage supplied.
    \begin{equation}\label{eq:kirchhoff}
        \sum V = \sum I R
    \end{equation}
    \begin{equation}\label{eq:kirchhoff2}
        \sum I = 0
    \end{equation}

    To determine if our measured values are accurate, we will compare them to the theoretical values using an agreement test.
    The agreement test is defined as
    \begin{equation}\label{eq:agreement_test}
        |R_{\text{mea}} - R_{\exp}| \le 2 \sqrt{\alpha^2_{\text{mea}} -
        \alpha^2_{\exp} }
    \end{equation}


    \section{Measuring Resistance (Part 0)}\label{sec:measuring-resistance}


    \subsection{Methods}\label{subsec:resistance_methods}

    This lab will use a digital multimeter (DMM) to measure the resistance across a 1 $\Omega$ resistor.
    The procedure was fairly simple.
    The DMM was set to measure resistance, and the banana cables were connected to the resistor.
    This can be seen in Figure~\ref{fig:resistance_setup}.

    \begin{figure}[h]
        \begin{annotatedFigure}
        {\includegraphics[width=1.0\linewidth]{resources/images/part0_setup}}
            \annotatedFigureBox{0.138,0.3873}{0.348,0.9773}{A}{0.138,0.3873}%bl
            \annotatedFigureBox{0.606,0.424}{0.736,0.5378}{B}{0.606,0.424}%bl
            \annotatedFigureBox{0.586,0.5419}{0.716,0.6907}{C}{0.586,0.5419}%bl
            \annotatedFigureBox{0.54,0.2497}{0.67,0.3983}{D}{0.54,0.2497}%bl
        \end{annotatedFigure}

        \caption{The DMM (A) measures the resistance across a 1 $\Omega$ resistor (B) via banana cables (C) and (D).}
        \label{fig:resistance_setup}
    \end{figure}

    \subsection{Analysis and Conclusion}\label{subsec:resistance_analysis}

    The resistance was measured to be 1.1 $\Omega$.
    The nominal value for the resistance was $1 \pm 1\%~\Omega$.
    The agreement test, from Equation~\ref{eq:agreement_test}, shows that the measured value is within the expected range.
    \begin{e}
        \begin{align*}
            |1.1 - 1| &\le 2 \sqrt{0.1^2 + 0.01^2} \\
            0.1 &\le 2 \sqrt{.0101} \\
            0.1 &\le .201
        \end{align*}
    \end{e}

    This considers the error in the measurement and the error in the expected value.
    The use of a DMM makes human error virtually nonexistent.
    In conclusion, the measured resistance is within the expected range, and the DMM is a reliable tool for measuring resistance.

    \section{Measuring Voltage (Experiment 1A)}\label{sec:voltage}

    \subsection{Methods}\label{subsec:voltage_methods}

    Figure~\ref{fig:current_setup_1a} shows the setup used in this experiment, consisting of a battery and 2 10,000 $\Omega$ resistors. We measured the voltage at 2 different parts of the circuit - the initial voltage from the DAC and the voltage after $R_1$. We then found the transfer ratio H, found from equation~\ref{eq:transfer_ratio}.

    \begin{figure}[h]
        \begin{center}
            \begin{circuitikz}[american]
                \draw (0,0) to[battery1=$V_1$, inverted] ++(6,0)
                -- ++(0,-3)
                to[R=$R_1$] ++(-3,0)
                to[R=$R_2$, *-*] ++(-3,0)
                -- ++(0,3);
            \end{circuitikz}
        \end{center}
<<<<<<< HEAD
        \caption {Abstract circuit configuration for measuring current. $R_1$ and $R_2$ are 10,000 $\Omega$ resistors.}
        \label{fig:current_setup_1a}
=======
        \caption {Abstract circuit configuration for measuring current. $R_1$ and $R_2$ are 1000 $\Omega$ resistors.}
        \label{fig:voltage_setup}
>>>>>>> 2a833de0c622e5bc8f151cf5e190e6e4b2f61ea5
    \end{figure}

    We used both the IOLab and the DMM to measure the voltage difference. We then compared them to see if they agreed.

    \subsection{Analysis}\label{subsec:voltage_analysis}

    When we used the IOLab to measure the voltage difference across $R_1$, we got the following graph:
    \todo{put in graph called Voltage over Time Part 1A1}

    This graph shows the input (A7) volts and the output (A8) volts. When we take the time segment from 4 seconds to 8 seconds, and use equation~\ref{eq:transfer_ratio}, we find the Transfer Ratio (H): 0.5006 ± 7.218*10\^-5. We can use a comparison test, using 1\% as the uncertainty for resistors greater than 1 ohm. The error for the transfer ratio is the error for \( V \), which is 0.01.
    
    The agreement test, from Equation~\ref{eq:agreement_test}, shows that the measured value is within the expected range.
    \begin{e}
        \begin{align*}
            |0.5 - 0.5006| &\le 2 \sqrt{0.01^2 + 0.00007218^2} \\
            0.0006 &\le 2 \sqrt{.0001} \\
            0.0006 &\le 0.02
        \end{align*}

    When we used the DMM to measure the voltage difference across $R_1$, we got a value of 1.0 V. We can use a comparison test, using 1\% as the uncertainty for resistors greater than 1 ohm. We can use the average value from the A8 voltage meter of 1.005 V to compare to the DMM value.
    
    The agreement test, from Equation~\ref{eq:agreement_test}, shows that the DMM and IOLab values agree.
    \begin{e}
        \begin{align*}
            |1.005 - 1| &\le 2 \sqrt{0.01^2 + 0.01^2} \\
            0.0006 &\le 2 \sqrt{.0002} \\
            0.0006 &\le 0.0282
        \end{align*}
    \end{e}

    \subsection{Conclusion}\label{subsec:voltage_conclusion}

    \section{Measuring Current (Experiment 1B)}\label{sec:current}

    \subsection{Methods}\label{subsec:current_methods}

    \begin{figure}[h]
        \begin{center}
            \begin{circuitikz}[american]
                \draw (0,0) to[battery1=$V_1$, inverted] ++(6,0)
                -- ++(0,-3)
                to[R=$R_1$] ++(-3,0)
                to[R=$R_2$, *-*] ++(-3,0)
                -- ++(0,3);
            \end{circuitikz}
        \end{center}
        \caption {Abstract circuit configuration for measuring current. $R_2$ is the 1 $\Omega$ resistor which it is measured across.}
        \label{fig:current_setup}
    \end{figure}

    \subsection{Analysis}\label{subsec:current_analysis}

    \subsection{Conclusion}\label{subsec:current_conclusion}


    \section{Ohmic Behavior (Experiment 2A)}\label{sec:ohmic}

    \subsection{Methods}\label{subsec:ohmic_methods}

    To verify Ohm's Law, we set up a circuit with a 10,000 $\Omega$ resistor.
    The IOLab was used to provide a voltage to the circuit.
    The DMM was used to measure the voltage across the resistor.
    This can be seen in Figure~\ref{fig:ohmic_setup}.
    The IOLab was set to provide a voltage ranging from 1 V to 3.3 V.

    \begin{figure}[h]
        \begin{annotatedFigure}
        {\includegraphics[width=1.0\linewidth]{resources/images/part2_setup}}
            \annotatedFigureBox{0.196,0.6112}{0.483,0.946}{A}{0.196,0.6112}%bl
            \annotatedFigureBox{0.642,0.6217}{0.7352,0.7262}{B}{0.642,0.6217}%bl
            \annotatedFigureBox{0.585,0.6934}{0.639,0.7834}{C}{0.585,0.6934}%bl
            \annotatedFigureBox{0.011,0.3155}{0.177,0.6714}{D}{0.011,0.3155}%bl
        \end{annotatedFigure}
        \caption{The IOLab (A) provides a voltage that runs through the breadboard. On the board a 10 k$\Omega$ resistor (B) is in series. The ground is connected via banana cables that go through the DMM (D).}
        \label{fig:ohmic_setup}
    \end{figure}
    
    First, we measured Ohmic behaviour.
    The DMM was set to measure voltage, and the banana cables were connected to the resistor.
    We recorded the voltage across the resistor as the voltage was increased.
    This resulted in the following data:
    
    \begin{figure}[h!]
        \centering
        \begin{tabular}{|c|c|}
            \hline
            \text{Volts (V)} & \text{Amps (A)} \\
            \hline
            0 & 0 \\
            0.3 & 0.00003 \\
            0.6 & 0.00006 \\
            1.0 & 0.00009 \\
            1.3 & 0.00013 \\
            1.6 & 0.00015 \\
            1.9 & 0.00020 \\
            2.2 & 0.00023 \\
            2.5 & 0.00026 \\
            2.8 & 0.00030 \\
            3.1 & 0.00032 \\
            \hline
        \end{tabular}
        \caption{Voltage and current data for the experiment.}
        \label{fig: ohmic_data}
    \end{figure}


    \subsection{Analysis}\label{subsec:ohmic_analysis}

    The data shows a linear relationship between the voltage and the current, as expected.
    
    \subsection{Conclusion}\label{subsec:ohmic_conclusion}




    \section{Non-Ohmic Behavior (Experiment 2B)}\label{sec:nonohmic}

    \subsection{Methods}\label{subsec:nonohmic_methods}

    We measured ohmic behavior for 2 different LEDs, one green and one red.
    The LED was connected in series with a 1 k$\Omega$ resistor.
    We recorded the voltage across the LED as the DAC voltage was increased.
    This resulted in the following data for the green LED:
    \begin{figure}[h!]
        \centering
        \begin{tabular}{|c|c|c|}
            \hline
            \text{A7 (Input) Voltage (V)} & \text{A8 (After LED) Voltage (V)} & \text{Current (A)} \\
            \hline
            1.0 & 0 & 0 \\
            1.3 & 0 & 0 \\
            1.6 & 0.005 & 0.00001 \\
            1.7 & 0.027 & 0.00003 \\
            1.9 & 0.17 & 0.00017 \\
            2.2 & 0.43 & 0.00043 \\
            2.4 & 0.624 & 0.00063 \\
            2.5 & 0.72 & 0.00072 \\
            2.8 & 0.91 & 0.00091 \\
            3.1 & 1.20 & 0.00121 \\
            3.3 & 1.37 & 0.00138 \\
            \hline
        \end{tabular}
        \caption{Voltage and current data for the green LED experiment.}
        \label{fig:non_ohmic_data_green}
    \end{figure}

    We also got the following data for the red LED:
    \begin{figure}[h!]
        \centering
        \begin{tabular}{|c|c|c|}
            \hline
            \text{A7 (Input) Voltage (V)} & \text{A8 (After LED) Voltage (V)} & \text{Current (A)} \\
            \hline
            0.8 & 0 & 0 \\
            1.0 & 0 & 0 \\
            1.3 & 0.004 & 0.000005 \\
            1.6 & 0.027 & 0.000023 \\
            1.8 & 0.18 & 0.00018 \\
            2.1 & 0.50 & 0.00054 \\
            2.3 & 0.799 & 0.0008 \\
            2.5 & 0.98 & 0.001 \\
            2.8 & 1.19 & 0.00123 \\
            3.1 & 1.41 & 0.00140 \\
            3.3 & 1.67 & 0.00165 \\
            \hline
        \end{tabular}
        \caption{Voltage and current data for the red LED experiment.}
        \label{fig:non_ohmic_data_red}
    \end{figure}

    
    \subsection{Analysis}\label{subsec:nonohmic_analysis}

    \subsection{Conclusion}\label{subsec:nonohmic_conclusion}





    \section{Kirchhoff's Laws}\label{sec:kirchoff}

    \subsection{Methods}\label{subsec:kirchoff_methods}

    \begin{figure}[h]
        \begin{center}
            \begin{circuitikz}[american]
                \draw (0,0) to[battery1=$V_1$] ++(0,3)
                to[R=$R_1$] ++(3,0) coordinate(P1)
                to[R=$R_3$, -*] ++(0,-3)
                node[above right] {$A$}
                -- (0,0);
                \draw (P1) to[R=$R_2$] ++(3,0)
                to[battery1, l=$V_2$] ++(0,-3) -- ++(-3,0);
            \end{circuitikz}
        \end{center}
        \caption {Abstract circuit configuration for Kirchhoff's Laws experiment.}
        \label{fig:kirchoff_setup}
    \end{figure}

    For this experiment, we set up a circuit with three resistors, $R_1$, $R_2$, and $R_3$, measuring 4.7k, 10k, and 4.7k $\Omega$, respectively.
    The goal was to measure the current before and after Junction $A$.
    To measure the current, we placed 1 $\Omega$ resistors before and after the junction.
    This can be seen in Figure~\ref{fig:kirchoff_setup}.
    We then used the DMM to measure the voltage across the resistors.
    The 1 $\Omega$ resistors added a negligible amount of resistance to the circuit, allowing them to act as ammeters.

    \subsection{Analysis}\label{subsec:kirchoff_analysis}

    Given figure~\ref{fig:kirchoff_setup}, we can use Kirchhoff's Laws to analyze the expected currents in the circuit.
    Taking $I_1$ as the initial current from the DAC ($V_1$) flowing into $R_1$, $I_2$ as the current after the batterie ($V_2$) flowing into $R_2$, and $I_3$ as the current after the resistor $R_3$, we can use the following equations:
    \begin{align*}
        Loop~1: V_1 - I_1 R_1 - I_2 R_2 &= 0 \\
        Loop~2: V_2 - I_2 R_2 - I_3 R_3 &= 0 \\
        Junction~A: I_3 - I_2 - I_1 &= 0
    \end{align*}
    These equations give:
    \begin{align*}
        I_1 &= \frac{V_1(R_2 + R_3) - V_2 R_3}{R_1 R_2 + R_2 R_3 + R_1 R_3} \\
        I_2 &= \frac{V_2(R_1 + R_3) - V_1 R_3}{R_1 R_2 + R_2 R_3 + R_1 R_3} \\
        I_3 &= \frac{V_1 R_2 + V_2 R_1}{R_1 R_2 + R_2 R_3 + R_1 R_3}
    \end{align*}
    For this experiment, we used:
    \begin{align*}
        V_1 &= 3.3 \pm 0.1 \text{ V} \\
        V_2 &= 4.5 \pm 0.1 \text{ V} \\
        R_1 &= 4.7k \pm 0.1k \text{ $\Omega$} \\
        R_2 &= 10k \pm 0.1k \text{ $\Omega$} \\
        R_1 &= 4.7k \pm 0.1k \text{ $\Omega$}
    \end{align*}
    which results in expected values for the currents of:
%    ['I1: 0.236 mA', 'I2: 0.231 mA', 'I3: 0.466 mA']
    \begin{align*}
        I_{1exp} &= 0.236 \pm 0.01 \text{mA} \\
        I_{2exp} &= 0.231 \pm 0.01 \text{mA} \\
        I_{3exp} &= 0.466 \pm 0.01 \text{mA}
    \end{align*}

    With the DMM, we measured the currents to be:
    \begin{align*}
        I_{1mea} &= 0.168 \pm 0.001 \text{mA} \\
        I_{2mea} &= 0.156 \pm 0.001 \text{mA} \\
        I_{3mea} &= 0.504 \pm 0.001 \text{mA}
    \end{}

    The agreement test, from Equation~\ref{eq:agreement_test}, shows that the measured values are not within the expected range.
    \begin{e}
        \begin{align*}
            |0.504 - 0.466| &\ge 2 \sqrt{0.001^2 - 0.01^2} \\
            0.038 &\ge 2 \sqrt{.0001} \\
            0.038 &\ge 0.0201
        \end{align*}
    \end{e}

    We also found that our data did not obey Kirchhoff's Junction rule, as we can clearly see that $I_1$ + $I_2$ does not equal $I_3$. This means our data was off significantly and we should not expect it to be within an acceptable range.

    \subsection{Conclusion}\label{subsec:kirchoff_conclusion}

    \todo{Explain the results of the agreement test}
    \todo{Consider sources of error}
    etc\ldots

    \section{Conclusion}\label{sec:conclusion}


    \section{References}\label{sec:references}

    Lab Manual

\end{document}