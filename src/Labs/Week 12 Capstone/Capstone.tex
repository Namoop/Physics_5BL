%! suppress = EscapeHashOutsideCommand
%! Author = Theodore Capinski
%! Date = 4/7/2024

% Preamble
\documentclass[11pt]{article}
\let\oldsection\section
\renewcommand\section{\clearpage\oldsection}
\setcounter{section}{-1}
\counterwithin{figure}{section}

% Packages
\usepackage{amsmath}
\usepackage{hyperref}
\usepackage{graphicx}
\usepackage{tikz}
\usepackage{indentfirst}
\usepackage{calc}
\usepackage{float}

%%%%%%%%%%%%%%%%%%%%%%%%%%%%%%%%%%%%%%%%%%%%%%%%%%%%%%%%%%%%%%%%%%%%%%
% LaTeX Overlay Generator - Annotated Figures v0.0.1
% Created with http://ff.cx/latex-overlay-generator/
%%%%%%%%%%%%%%%%%%%%%%%%%%%%%%%%%%%%%%%%%%%%%%%%%%%%%%%%%%%%%%%%%%%%%%
%\annotatedFigureBoxCustom{bottom-left}{top-right}{label}{label-position}{box-color}{label-color}{border-color}{text-color}
\newcommand*\annotatedFigureBoxCustom[8]{\draw[#5,thick,rounded corners] (#1) rectangle (#2);\node at (#4) [fill=#6,thick,shape=circle,draw=#7,inner sep=2pt,font=\sffamily,text=#8] {\textbf{#3}};}
%\annotatedFigureBox{bottom-left}{top-right}{label}{label-position}
\newcommand*\annotatedFigureBox[4]{\annotatedFigureBoxCustom{#1}{#2}{#3}{#4}{white}{white}{black}{black}}
\newcommand*\annotatedFigureText[4]{\node[draw=none, anchor=south west, text=#2, inner sep=0, text width=#3\linewidth,font=\sffamily] at (#1){#4};}
\newenvironment {annotatedFigure}[1]{\centering\begin{tikzpicture}
                                                   \node[anchor=south west,inner sep=0] (image) at (0,0) { #1};\begin{scope}[x={(image.south east)},y={(image.north west)}]}{\end{scope}\end{tikzpicture}}
%%%%%%%%%%%%%%%%%%%%%%%%%%%%%%%%%%%%%%%%%%%%%%%%%%%%%%%%%%%%%%%%%%%%%%

\newcommand{\todo}[1]{\textcolor{red}{TODO: #1}\PackageWarning{TODO:}{#1!}}

\title{Physics 5BL Lab Report Capstone Project}
\author{T.~Capinski, A.~Patel \and B. Davis}

% Document
\begin{document}
    \maketitle
    \tableofcontents

    \section*{Introduction}\label{sec:introduction}
    \addcontentsline{toc}{section}{Introduction}

    In this report, we will be discussing the results of our capstone project.
    The project was split into three parts.
    The first part was to test different variables in a solenoid to see how they affect the magnetic field.
    We relied on the formula
    \begin{equation}
        \Epsilon = -N A \frac{d\Phi}{dt}
        \label{eq:emf}
    \end{equation}
    to predict how the magnetic field would change.
    The second part was to create our own solenoid and test it.
    The third part was to add an iron core to the solenoid to see how it affected the magnetic field.

    We hung a magnet from a spring to create a changing magnetic field.
    This led to an oscillating current being induced in a solenoid.
    We measured the peak induced current for each setup for consistency.

    First, we set up the solenoid with a constant number of turns and one magnet.
    This gave us a baseline for how the solenoid would behave.

    Next, we doubled the number of turns by adding another solenoid in series.
    Based on Equation~\ref{eq:emf}, we expected the peak current to double.
    The result was a voltage $1.92 \times$ as strong as the original setup.
    This agreed with our prediction.

    We then tested how the number of magnets affected the peak current.
    We doubled the number of magnets, so based on Equation~\ref{eq:emf}, we expected the peak current to double.
    The result was a voltage $3.74 \times$ as strong as the original setup.

    We then added mass to the spring, which increased the maximum speed of the magnet, and therefore the rate of change of the magnetic field.
    This led to a voltage $3.26 \times$ as strong as the original setup.

    Finally, we built our own solenoids.
    We did this because we could not vary the radius of the solenoid in the lab.
    We built one with a 3cm radius and one with a 6cm radius.
    We expected the peak current to be $4 \times$ as strong for the 6cm radius solenoid.
    The result was a voltage $4.29 \times$ as strong in the larger solenoid.

    We then added an iron core to the solenoid.
    We expected the peak current to be stronger, but we didn't know by how much.
    The result was a voltage $z \times$ as strong as the original setup.
    We also weren't sure of the exact material of the core, so with this number and some research, we determined it was likely steel.

    Overall, we found how the number of turns, the number of magnets, the mass on the spring, the radius of the solenoid, and the material of the core all affected the peak current.
    Most of our results were consistent with our predictions, but there were some discrepancies.

    \section*{Theory}\label{sec:theory}
    \addcontentsline{toc}{section}{Theory}


    \section{Part 1: Testing different variables}\label{sec:part_1}
    \subsection{Methods}\label{subsec:part_1_methods}
    \subsection{Analysis}\label{subsec:part_1_analsysis}
    \subsection{Conclusion}\label{subsec:part_1_conclusion}

    \section{Part 2: Creating our own solonoid }\label{sec:part_2}
    \subsection{Methods}\label{subsec:part_2_methods}
    \subsection{Analysis}\label{subsec:part_2_analsysis}
    \subsection{Conclusion}\label{subsec:part_2_conclusion}

    \section{Lab Conclusion}\label{sec:lab_conclusion}



    \appendix
    \section{References}\label{sec:references}

    Lab Manual
\end{document}