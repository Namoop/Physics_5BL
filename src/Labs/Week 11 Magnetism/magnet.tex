%! suppress = EscapeHashOutsideCommand
%! Author = Theodore Capinski
%! Date = 4/7/2024

% Preamble
\documentclass[11pt]{article}
\let\oldsection\section
\renewcommand\section{\clearpage\oldsection}
\setcounter{section}{-1}
\counterwithin{figure}{section}

% Packages
\usepackage{amsmath}
\usepackage{hyperref}
\usepackage{graphicx}
\usepackage{tikz}
\usepackage{indentfirst}
\usepackage{calc}
\usepackage{float}

%%%%%%%%%%%%%%%%%%%%%%%%%%%%%%%%%%%%%%%%%%%%%%%%%%%%%%%%%%%%%%%%%%%%%%
% LaTeX Overlay Generator - Annotated Figures v0.0.1
% Created with http://ff.cx/latex-overlay-generator/
%%%%%%%%%%%%%%%%%%%%%%%%%%%%%%%%%%%%%%%%%%%%%%%%%%%%%%%%%%%%%%%%%%%%%%
%\annotatedFigureBoxCustom{bottom-left}{top-right}{label}{label-position}{box-color}{label-color}{border-color}{text-color}
\newcommand*\annotatedFigureBoxCustom[8]{\draw[#5,thick,rounded corners] (#1) rectangle (#2);\node at (#4) [fill=#6,thick,shape=circle,draw=#7,inner sep=2pt,font=\sffamily,text=#8] {\textbf{#3}};}
%\annotatedFigureBox{bottom-left}{top-right}{label}{label-position}
\newcommand*\annotatedFigureBox[4]{\annotatedFigureBoxCustom{#1}{#2}{#3}{#4}{white}{white}{black}{black}}
\newcommand*\annotatedFigureText[4]{\node[draw=none, anchor=south west, text=#2, inner sep=0, text width=#3\linewidth,font=\sffamily] at (#1){#4};}
\newenvironment {annotatedFigure}[1]{\centering\begin{tikzpicture}
                                                   \node[anchor=south west,inner sep=0] (image) at (0,0) { #1};\begin{scope}[x={(image.south east)},y={(image.north west)}]}{\end{scope}\end{tikzpicture}}
%%%%%%%%%%%%%%%%%%%%%%%%%%%%%%%%%%%%%%%%%%%%%%%%%%%%%%%%%%%%%%%%%%%%%%

\newcommand{\todo}[1]{\textcolor{red}{TODO: #1}\PackageWarning{TODO:}{#1!}}

\title{Physics 5BL Lab Report Faradays Law}
\author{T.~Capinski \and A.~Patel}

% Document
\begin{document}
    \maketitle
    \tableofcontents

    \section*{Introduction}\label{sec:introduction}
    \addcontentsline{toc}{section}{Introduction}

    In this lab we did 3 experiments to verify Faradays Law. 
    
    In the first part, we created a generator using the Earth's magnetic field. 
    
    In the second part, we verified Faradays Law by creating a solenoid and measuring the induced current. This part was selected as a summary section and will be mentioned again briefly at the end of this report.
    
    In the third part, we built a motor using the Earth's magnetic field.


    \section*{Theory}\label{sec:theory}
    \addcontentsline{toc}{section}{Theory}


    \section{Part 1: Creating a Generator with the Earth’s Field}\label{sec:part_1}
    \subsection{Methods}\label{subsec:part_1_methods}
    \subsection{Analysis}\label{subsec:part_1_analsysis}
    \subsection{Conclusion}\label{subsec:part_1_conclusion}

    \section{Part 3: Building a Motor }\label{sec:part_3}
    \subsection{Methods}\label{subsec:part_3_methods}
    \subsection{Analysis}\label{subsec:part_3_analsysis}
    \subsection{Conclusion}\label{subsec:part_3_conclusion}

    \section{Lab Conclusion}\label{sec:lab_conclusion}


    

    \section{Part 2: Verifying Faraday’s Law (summary section)}\label{sec:part_2}

    For this part, we verified Faradays Law by creating a "solenoid" out of wire and measuring the magnetic field and induced current in the coil. To induce a current, we needed a changing B-field, achieved by moving a magnet up and down directly vertical to the center of the coil, which was placed right on top of the IOlabs magnetar. The setup is shown in the following picture:

    todo: setup part 2
    caption: part 2 setup

    We then plotted the high gain vs the magnetometer and got the following graph:

    todo: high gain vs magnetometer graph
    caption: part 2 graph of high gain (mv) vs magnetometer (µT)


    \appendix
    \section{References}\label{sec:references}

    Lab Manual
\end{document}