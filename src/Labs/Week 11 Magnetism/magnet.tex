%! suppress = EscapeHashOutsideCommand
%! Author = Theodore Capinski
%! Date = 4/7/2024

% Preamble
\documentclass[11pt]{article}
\let\oldsection\section
\renewcommand\section{\clearpage\oldsection}
\setcounter{section}{-1}
\counterwithin{figure}{section}

% Packages
\usepackage{amsmath}
\usepackage{hyperref}
\usepackage{graphicx}
\usepackage{tikz}
\usepackage{indentfirst}
\usepackage{calc}
\usepackage{float}

%%%%%%%%%%%%%%%%%%%%%%%%%%%%%%%%%%%%%%%%%%%%%%%%%%%%%%%%%%%%%%%%%%%%%%
% LaTeX Overlay Generator - Annotated Figures v0.0.1
% Created with http://ff.cx/latex-overlay-generator/
%%%%%%%%%%%%%%%%%%%%%%%%%%%%%%%%%%%%%%%%%%%%%%%%%%%%%%%%%%%%%%%%%%%%%%
%\annotatedFigureBoxCustom{bottom-left}{top-right}{label}{label-position}{box-color}{label-color}{border-color}{text-color}
\newcommand*\annotatedFigureBoxCustom[8]{\draw[#5,thick,rounded corners] (#1) rectangle (#2);\node at (#4) [fill=#6,thick,shape=circle,draw=#7,inner sep=2pt,font=\sffamily,text=#8] {\textbf{#3}};}
%\annotatedFigureBox{bottom-left}{top-right}{label}{label-position}
\newcommand*\annotatedFigureBox[4]{\annotatedFigureBoxCustom{#1}{#2}{#3}{#4}{white}{white}{black}{black}}
\newcommand*\annotatedFigureText[4]{\node[draw=none, anchor=south west, text=#2, inner sep=0, text width=#3\linewidth,font=\sffamily] at (#1){#4};}
\newenvironment {annotatedFigure}[1]{\centering\begin{tikzpicture}
                                                   \node[anchor=south west,inner sep=0] (image) at (0,0) { #1};\begin{scope}[x={(image.south east)},y={(image.north west)}]}{\end{scope}\end{tikzpicture}}
%%%%%%%%%%%%%%%%%%%%%%%%%%%%%%%%%%%%%%%%%%%%%%%%%%%%%%%%%%%%%%%%%%%%%%

\newcommand{\todo}[1]{\textcolor{red}{TODO: #1}\PackageWarning{TODO:}{#1!}}

\title{Physics 5BL Lab Report Faradays Law}
\author{T.~Capinski \and A.~Patel}

% Document
\begin{document}
    \maketitle
    \tableofcontents

    \section*{Introduction}\label{sec:introduction}
    \addcontentsline{toc}{section}{Introduction}

    In this lab we did 3 experiments to verify Faradays Law. 
    
    In the first part, we created a generator using the Earth's magnetic field. todo part 1 results
    
    In the second part, we verified Faradays Law by creating a solenoid and measuring the induced current. This part was selected as a summary section and will be mentioned again briefly later in the report.
    
    In the third part, we built a motor using the Earth's magnetic field. This part was selected as a summary section and will be mentioned again briefly later in the report.


    \section*{Theory}\label{sec:theory}
    \addcontentsline{toc}{section}{Theory}


    \section{Part 1: Creating a Generator with the Earth’s Field}\label{sec:part_1}
    \subsection{Methods}\label{subsec:part_1_methods}
    \subsection{Analysis}\label{subsec:part_1_analsysis}
    \subsection{Conclusion}\label{subsec:part_1_conclusion}


    \section{Part 2: Verifying Faraday’s Law (summary section)}\label{sec:part_2}

    For this part, we verified Faradays Law by creating a "solenoid" out of wire and measuring the magnetic field and induced current in the coil. To induce a current, we needed a changing B-field, achieved by moving a magnet up and down directly vertical to the center of the coil, which was placed right on top of the IOlabs magnetar. The setup is shown in the following picture:

    todo: setup part 2
    caption: part 2 setup

    We then plotted the high gain vs the magnetometer and got the following graph:

    todo: high gain vs magnetometer graph
    caption: part 2 graph of high gain (mv) vs magnetometer (µT)

    \section{Part 3: Building a Motor (summary section)}\label{sec:part_3}
    In this part, we attempted to create a motor that transforms electrical energy into kinetic energy. This motor uses a stationary magnet that acts as our magnetic field. We then set two conductors on either side of the magnet and a coil of wire with terminals that can reach out and rest in the conductors. This allows current to flow while also allowing the wire coil to rotate. We then connected one side of a battery to each conductor. 

    One of the most important parts of this setup was to only strip part of the coil. On one side, all of the coil needed to be stripped. However, on the other, only one half should be stripped in order to prevent the coil from decelerating itself during the other half-induced current. This took a lot of work and playing around with, but we finally achieved the following setup:

    todo: setup pic part 3
    caption: Setup for part 3

    Using this setup, we were able to take the following video of our motors motion:

    todo: faraday vid 1
    caption: video of motion of built motor

    We then were able to change the direction of the motor by switching the terminals the battery was connected to, and increase the speed by moving the coil closer to the magnet, as shown in the following video:

    todo: faraday vid 2
    caption: video of motion of built motor with increased speed and revered direction

    From these experiments, we can see that by carefully arranging the setup and adjusting parameters such as the position of the coil and the direction of the current flow, we were able to successfully convert electrical energy into kinetic energy. The motion of the motor demonstrates the principles of electromagnetic induction, where the changing magnetic field induces a current in the coil, resulting in rotational motion.

    The ability to change the direction of the motor by switching the battery terminals and increasing the speed of the motor by moving the coil closer to the magnet allows for practical application. This versatility allows for practical applications in various systems where reversible and precise motion is required. Fine-tuning such parameters can lead to optimization and efficiency improvements in motor design.
    
    Overall, these experiments showcase the fundamental concepts of electromagnetism and provide valuable insights into the operation and control of electric motors, laying the groundwork for further exploration and innovation in this field.







    \section{Lab Conclusion}\label{sec:lab_conclusion}


    \appendix
    \section{References}\label{sec:references}

    Lab Manual
\end{document}